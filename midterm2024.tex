\documentclass[10pt,addpoints,portrait]{exam}
\usepackage[UTF8]{ctex}
\usepackage{amsfonts,amssymb,amsmath, amsthm}
\usepackage{graphicx}
\usepackage{systeme}
\usepackage{pgf,tikz,pgfplots}
\usepackage{mathrsfs}
\usepackage{multicol}
\begin{document}
		\begin{center}
			\textbf{《概率论与数理统计》期中试卷} \\
			2024/2025学年第一学期\hspace{0.3cm}院系:\underline{\hspace{1in}}\hspace{0.3cm} 任课老师:\underline{\hspace{1in}} \\
			学号:\underline{\hspace{1in}}\hspace{0.3cm} 姓名:\underline{\hspace{1in}}\hspace{0.3cm} 考试成绩:\underline{\hspace{1in}}
			{	\small
				\begin{tabular}{|c|c|c|c|c|c|c|c|c|c|}
					\hline
					题号 & 1(8分) &2(8分)&3(8分) & 4(16分)& 5(16分) & 6(25分) & 7(10分) & 8(9分) & 总分 \\
					\hline
					得分 &  &  & & &  &  &  &  &  \\
					\hline
				\end{tabular}
			}
		\end{center} 
		
		\begin{questions} %------------------------------------------
			\question[8]
			设随机变量$X\sim N(\mu, 36), P(X < 15) = P(X > 35) = 0.048$。另外$\Phi(2) = 0.977$。试求$\mu, c$,使得$P(|X - \mu| \leq c) = 0.954$。
			\vspace{3cm}
			\question
			供应商A的豆种发芽率为95\%,供应商B的豆种发芽率为70\%。一个豆种包装公司的豆种30\%来自供应商A,70\%来自供应商B。并且全部豆种被均匀混合。
			\begin{parts}
				\part[4]
				计算从混合好的豆种中随机抽取的一粒可以发芽的概率$P(G)$.\\
				\vspace{0.5cm}
				\part[4]
				给定一粒被抽取的种子可以发芽,计算它来自供应商A的条件概率。\\
				\vspace{0.5cm}
			\end{parts}
			
			\question
			一个教授给她的学生们出了15道作业题,其中4道将被当作期中考试题。一个学生只有时间学习15道题中的5道。请计算如下概率
			\begin{parts}
				\part[4]
				学生学习到了3道期中考试题。\\
				\vspace{1.5cm}
				\part[4]
				至少学习到了1道期中考试题。\\
				\vspace{1.5cm}
			\end{parts}
			\question[10]
			$X$有密度函数$p(x) = x\exp(-x ^ 2 / 2)$, $0 < x < \infty$. $Y = X ^ 2$。计算$Y$的密度函数。\\
			\vspace{4cm}
			\question
			对于下面每个函数,(i)计算常量$c$使得$p(x)$是一个随机变量$X$的密度函数,(ii)计算概率分布函数$F(x) = P(X \leq x)$,(iii)计算$X$均值$EX$和方差$DX$
			\begin{parts}
				\part[8]
				$p(x) = (3 / 16)x ^ 2 , -c < x < c$。\\
				\vspace{5cm}
				\part[8]
				$p(x) = c / \sqrt{x}, 0 < x < 1$。\\
				\vspace{7cm}
			\end{parts}
			\question
			投掷两枚骰子,一个红色,一个黑色。每一个骰子等可能出现1、2、3、4、5、6六种结果。用$X$代表红色骰子出现的结果除以3的余数,用$Y$代表黑色骰子出现的结果除以3的余数。$Z = X + Y$。
			\begin{parts}
				\part[4]
				计算$X,Z$的联合分布列。\\
				\vspace{4cm}
				\part[4]
				计算$X$的边缘分布列。\\
				\vspace{2cm}
				\part[4]
				计算$Z$的边缘分布列。\\
				\vspace{2cm}
				\part[4]
				请说明$X, Z$是否独立,及其原因。\\
				\vspace{1cm}
			\end{parts}
			\question
			$X, Y$有联合密度函数$p(x, y) = x + y, 0 \leq x \leq 1, 0 \leq y \leq 1$
			\begin{parts}
				\part[5]
				计算边缘密度函数$p_X(x), p_Y(y)$。\\
				\vspace{2cm}
				\part[5]
				随机变量$X, Y$是否独立?请解释原因。\\
				\vspace{1.5cm}
				\part[5]
				计算$X, Y$的均值和方差。\\
				\vspace{6cm}
				\part[5]
				计算$P(X \leq Y)$ \\
				\vspace{2cm}
				\part[5]
				计算$Z = X + Y$的密度函数。\\
			\end{parts}
			\vspace{8cm}
			\question
			随机变量$X, Y$有联合分布列$p(x, y) = 1 / 4, (x, y) \in S = \{(0, 0), (1, 1), (1, -1), (2, 0)\}$
			\begin{parts}
				\part[4]
				$X, Y$是否独立?请解释原因。 \\
				\vspace{2cm}
				\part[5]
				计算协方差和相关系数。\\
				\vspace{3cm}
			\end{parts}
		\end{questions}
\end{document}