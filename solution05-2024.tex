%!TEX program = xelatex
\documentclass[14pt]{scrartcl} % A4 paper and 11pt font size
\usepackage[UTF8]{ctex}
\usepackage{lmodern}
\usepackage{amsmath}
\usepackage{amsfonts}
\usepackage{amssymb}
\usepackage[T1]{fontenc} % Use 8-bit encoding that has 256 glyphs
\usepackage{fourier} % Use the Adobe Utopia font for the document - comment this line to return to the LaTeX default
\usepackage[english]{babel} % English language/hyphenation
\usepackage{amsmath,amsfonts,amsthm} % Math packages
\usepackage{graphicx}
\usepackage{epstopdf}
\epstopdfsetup{outdir=./}
\usepackage{float}
\usepackage{lipsum} % Used for inserting dummy 'Lorem ipsum' text into the template
\usepackage{enumerate}
\usepackage{fancyvrb}
\usepackage{sectsty} % Allows customizing section commands
\allsectionsfont{\centering \normalfont\scshape} % Make all sections centered, the default font and small caps



\numberwithin{equation}{section} % Number equations within sections (i.e. 1.1, 1.2, 2.1, 2.2 instead of 1, 2, 3, 4)
\numberwithin{figure}{section} % Number figures within sections (i.e. 1.1, 1.2, 2.1, 2.2 instead of 1, 2, 3, 4)
\numberwithin{table}{section} % Number tables within sections (i.e. 1.1, 1.2, 2.1, 2.2 instead of 1, 2, 3, 4)

\setlength\parindent{0pt} % Removes all indentation from paragraphs - comment this line for an assignment with lots of text

%----------------------------------------------------------------------------------------
%	TITLE SECTION
%----------------------------------------------------------------------------------------

\newcommand{\horrule}[1]{\rule{\linewidth}{#1}} % Create horizontal rule command with 1 argument of height
\newcommand*{\Scale}[2][4]{\scalebox{#1}{$#2$}}%
\title{	
	\normalfont \huge
	\textsc{概率统计} \\ [25pt] % Your university, school and/or department name(s)
	\horrule{0.5pt} \\[0.4cm] % Thin top horizontal rule
	\huge 第五、六周作业答案 \\ % The assignment title
	\horrule{0.5pt} \\[0.4cm] % Thick bottom horizontal rule
	\date{}
}

\begin{document}
	\maketitle % Print the title
	习题二
	20. 
	\begin{center}
		\begin{tabular}{l|l|l|l}
			& $y=-1$ & $Y=1$ & $y=-2$ \\
			\hline
			$x=-1$ & $\frac{5}{20}(-2,1,1,-1)$ & $\frac{2}{20}(0,-1,-1,1)$ & $\frac{6}{20}\left(1,-2,-\frac{1}{2}, 2\right)$ \\
			\hline
			$x=2$ & $\frac{3}{20}(1,-2,-2,2)$ & $\frac{3}{20}(3,2,2,2)$ & $\frac{1}{20}(4,4,1,2)$ \\
			\hline
		\end{tabular}
	\end{center}
	
	(1)
	
	\begin{tabular}{c|c|c|c|c|c}
		$z$ & -2 & 0 & 1 & 3 & 4 \\
		\hline
		$p$ & $\frac{5}{20}$ & $\frac{2}{20}$ & $\frac{9}{20}$ & $\frac{3}{20}$ & $\frac{1}{20}$ \\
		\hline
	\end{tabular}
	
	(2) \begin{tabular}{l|l|l|l|l|l}
		$z$ & -2 & -1 & 1 & 2 & 4 \\
		\hline
		$p$ & $\frac{9}{20}$ & $\frac{2}{20}$ & $\frac{5}{20}$ & $\frac{3}{20}$ & $\frac{1}{20}$ \\
		\hline
	\end{tabular}
	
	(3) \begin{tabular}{c|c|c|c|c|c}
		$z$ & -2 & -1 & $-\frac{1}{2}$ & 1 & 2 \\
		\hline
		$p$ & $\frac{3}{20}$ & $\frac{2}{20}$ & $\frac{6}{20}$ & $\frac{6}{20}$ & $\frac{3}{20}$ \\
		\hline
	\end{tabular}
	
	(4) \begin{tabular}{c|c|c|c}
		$z$ & -1 & 1 & 2 \\
		\hline
		$p$ & $\frac{5}{20}$ & $\frac{2}{20}$ & $\frac{13}{20}$ \\
		\hline
	\end{tabular}
	
	
	23.
	
	\begin{center}
		\begin{tabular}{l|l|l|l}
			& $Y=1$ & $Y=2$ & $Y=3$ \\
			\hline
			$X=1$ & $(1,1)$ & $(2,1)$ & $(3,1)$ \\
			\hline
			$X=2$ & $(2,1)$ & $(2,2)$ & $(3,2)$ \\
			\hline
			$X=3$ & $(3,1)$ & $(3,2)$ & $(3,3)$ \\
			\hline
		\end{tabular}
	\end{center}
	
	\begin{center}
		\begin{tabular}{c|c|c|c}
			& $V=1$ & $V=2$ & $V=3$ \\
			\hline
			$U=1$ & $\frac{1}{9}$ & 0 & 0 \\
			\hline
			$U=2$ & $\frac{2}{9}$ & $\frac{1}{9}$ & 0 \\
			\hline
			$U=3$ & $\frac{2}{9}$ & $\frac{2}{9}$ & $\frac{1}{9}$ \\
			\hline
		\end{tabular}
	\end{center}
	习题三
	2.
	
	$$
	\begin{aligned}
		p(Y=k \mid X=n) & =\frac{1}{n}, k=1,2, \cdots, n \\
		p(X=n)=\frac{1}{3}, & \\
		p(X=n, Y=k) & =p(Y=k \mid X=n) p(X=n) \\
		& =\frac{1}{n} \times \frac{1}{3} \\
		& =\frac{1}{3 n}, \quad n=1,2,3, k=1, \cdots, n .
	\end{aligned}
	$$
	\begin{center}
		\begin{tabular}{c|c|c|c|c}
			& $y=1$ & $y=2$ & $y=3$ & $p_{i}$. \\
			\hline
			$x=1$ & $\frac{1}{3}$ & 0 & 0 & $\frac{1}{3}$ \\
			\hline
			$x=2$ & $\frac{1}{6}$ & $\frac{1}{6}$ & 0 & $\frac{1}{3}$ \\
			\hline
			$x=3$ & $\frac{1}{9}$ & $\frac{1}{9}$ & $\frac{1}{9}$ & $\frac{1}{3}$ \\
			\hline
			$p \cdot j$ & $\frac{11}{18}$ & $\frac{15}{18}$ & $\frac{1}{9}$ & 1 \\
			\hline
		\end{tabular}
	\end{center}
	
	4.见例3.3。
	
	9.
	
	(1)
	
	$$
	\begin{aligned}
		& \int_{0}^{3} \int_{0}^{3} p(x, y) d x d y\\
		= &\int_{0}^{3} \int_{0}^{3} A x^{2} y^{2} d x d y \\
		= & A \int_{0}^{3} x^{2} d x \int_{0}^{3} y^{2} d y \\
		= & A \times 9 \times 9 \\
		= & 81 \times A \\
		= & 1
	\end{aligned}
	$$
	
	$$
	\begin{aligned}
		& \Rightarrow A=\frac{1}{81} \\
		& p_{X}(x)=\frac{1}{9} x^{2}, x \in(0,3) \\
		& p_{Y}(y)=\frac{1}{9} y^{2}, \quad y \in(0,3) \\
		& p(x, y)=p_{X}(x) p_{Y}(y)=\frac{1}{81} x^{2} y^{2} \Rightarrow X, Y \text {独立 }
	\end{aligned}
	$$
	(2)
	$$
	\begin{aligned}
		& \int_{0}^{3} \int_{0}^{y} A x^{2} y^{2} d x d y \\
		= & A \int_{0}^{3} y^{2} \cdot \frac{y^{3}}{3} d y \\
		= & \frac{A}{3} \int_{0}^{3} y^{5} d y \\
		= & \frac{A}{3} \int_{0}^{3} \frac{1}{6} d y^{6} \\
		= & \frac{A}{18} \cdot 3^{6} \\
		= & \frac{81}{2} A \\
		= & 1 \\
		\Rightarrow & A=\frac{2}{81}
	\end{aligned}
	$$
	
	由于Y的取值影响X的取值范围,所以X. Y 不独。
	
	11.(1)
	$$
	\begin{aligned}
		p_{x}(x) & =\int_{x}^{\infty} p(x, y) d y \\
		& =\int_{x}^{\infty} e^{-y} d y=e^{-x}, x \in(0, \infty)
	\end{aligned}
	$$
	
	$$
	\begin{aligned}
		&p_Y(y)=\int_{0}^{y} p(x, y) d x \\
		& =\int_{0}^{y} e^{-y} d x \\
		& =y e^{-y}, y \in(0, \infty)
	\end{aligned}
	$$
	
	(2)
	
	\begin{align}
		&P(x+Y<1)=\int_{0}^{\frac{1}{2}} \int_{x}^{1-x} e^{-y} d y d x \\
		&=1+e^{-1}-2 e^{-\frac{1}{2}}
	\end{align}
	20. 
	\begin{center}
		\begin{tabular}{l|l|l|l}
			& $y=-1$ & $Y=1$ & $y=-2$ \\
			\hline
			$x=-1$ & $\frac{5}{20}(-2,1,1,-1)$ & $\frac{2}{20}(0,-1,-1,1)$ & $\frac{6}{20}\left(1,-2,-\frac{1}{2}, 2\right)$ \\
			\hline
			$x=2$ & $\frac{3}{20}(1,-2,-2,2)$ & $\frac{3}{20}(3,2,2,2)$ & $\frac{1}{20}(4,4,1,2)$ \\
			\hline
		\end{tabular}
	\end{center}
	
	(1)
	
	\begin{tabular}{c|c|c|c|c|c}
		$z$ & -2 & 0 & 1 & 3 & 4 \\
		\hline
		$p$ & $\frac{5}{20}$ & $\frac{2}{20}$ & $\frac{9}{20}$ & $\frac{3}{20}$ & $\frac{1}{20}$ \\
		\hline
	\end{tabular}
	
	(2) \begin{tabular}{l|l|l|l|l|l}
		$z$ & -2 & -1 & 1 & 2 & 4 \\
		\hline
		$p$ & $\frac{9}{20}$ & $\frac{2}{20}$ & $\frac{5}{20}$ & $\frac{3}{20}$ & $\frac{1}{20}$ \\
		\hline
	\end{tabular}
	
	(3) \begin{tabular}{c|c|c|c|c|c}
		$z$ & -2 & -1 & $-\frac{1}{2}$ & 1 & 2 \\
		\hline
		$p$ & $\frac{3}{20}$ & $\frac{2}{20}$ & $\frac{6}{20}$ & $\frac{6}{20}$ & $\frac{3}{20}$ \\
		\hline
	\end{tabular}
	
	(4) \begin{tabular}{c|c|c|c}
		$z$ & -1 & 1 & 2 \\
		\hline
		$p$ & $\frac{5}{20}$ & $\frac{2}{20}$ & $\frac{13}{20}$ \\
		\hline
	\end{tabular}
	
	
	23.
	
	\begin{center}
		\begin{tabular}{l|l|l|l}
			& $Y=1$ & $Y=2$ & $Y=3$ \\
			\hline
			$X=1$ & $(1,1)$ & $(2,1)$ & $(3,1)$ \\
			\hline
			$X=2$ & $(2,1)$ & $(2,2)$ & $(3,2)$ \\
			\hline
			$X=3$ & $(3,1)$ & $(3,2)$ & $(3,3)$ \\
			\hline
		\end{tabular}
	\end{center}
	
	\begin{center}
		\begin{tabular}{c|c|c|c}
			& $V=1$ & $V=2$ & $V=3$ \\
			\hline
			$U=1$ & $\frac{1}{9}$ & 0 & 0 \\
			\hline
			$U=2$ & $\frac{2}{9}$ & $\frac{1}{9}$ & 0 \\
			\hline
			$U=3$ & $\frac{2}{9}$ & $\frac{2}{9}$ & $\frac{1}{9}$ \\
			\hline
		\end{tabular}
	\end{center}
	
	27. $F_{z}(z)=P(Z \leqslant z)=P(X - Y \leqslant z)=1-P(X-Y>Z)$
	
	$$
	\begin{aligned}
		& =1-\int_{z}^{1} \int_{0}^{x-z} 3 x d y d x \\
		& =\frac{3}{2} z-\frac{1}{2} z^{3}, z \in(0,1)
	\end{aligned}
	$$
	
	$p_{z}(z)=F_{z}^{\prime}(z)=\frac{3}{2}-\frac{3}{2} z^{2}, z \in(0,1)$
	
	30. $p(x=k)=\frac{\lambda^{k}}{k !} e^{-\lambda}$
	
	$$
	\frac{p(x=1)}{p(x=2)}=\frac{2}{\lambda}=2 \Rightarrow \lambda=1
	$$
	
	$p(Z>0)=1-p(Z \leq 0)=1-p(x \leq 0, y \leq 0)=1-\left(e^{-1}\right)^{2}=1 - e^{-2}$

32.
\[
\begin{aligned}
	& p_X(x)=\alpha e^{-\alpha x}, \quad x>0 \\
	& F_X(x)=1-e^{-\alpha x}, \quad x>0
\end{aligned}
\]
(1)
\[
\begin{aligned}
	Z&=\min (X, Y) \\
	F_Z(z)&=1 -  \left(1-F_X(z)\right)\left(1-F_Y(z)\right) \\
	& =1-e^{-(\alpha+\beta) z}, z>0 \\
	p_Z(z)&=F_Z^{\prime}(z)=(\alpha+\beta) e^{-(\alpha+\beta) z}, z>0
\end{aligned}
\]
(2)
\[
\begin{aligned}
	& Z=\max (X, Y) \\
	& F_Z(z)=F_X(z) F_Y(z)=\left(1-e^{-\alpha z}\right)\left(1-e^{-\beta z}\right), z, 0 \\
	& p_Z(z)=F_Z^{\prime}(z)=\alpha e^{-\alpha z}+\beta e^{-\beta z}-(\alpha+\beta) e^{-(\alpha + \beta) \beta z}, z>0
\end{aligned}
\]
(3)
\[
\begin{aligned}
	Z& =X+Y\\
	p_Z(z) & =\int_{-\infty}^{\infty} p_x(x) p_y(z-x) d x \\
	& =\int_0^z \alpha e^{-\alpha x} \beta e^{-\beta(z-x)} d x \\
	& =\frac{\alpha \beta}{\beta-\alpha}\left(e^{-\alpha z}-e^{-\beta z}\right), z>0
\end{aligned}
\]
33.
\[
\begin{aligned}
	F_Z(z) & =P(X=1) p(Z\leq z \mid X=1)+p(X=2) p(Z \leq z \mid X=2) \\
	& =P(X=1) P(2 X+Y \leq z \mid X=1)+p(X=2) p(2X+Y \leq z \mid X=2) \\
	& =\frac{1}{3} P(Y \leq z-2 \mid X=1)+\frac{2}{3} p(Y \leq z-4 \mid X=2) \\
	& =\frac{1}{3} P(Y \leq z-2)+\frac{2}{3} p(Y \leq z-4) \\
	& =\frac{1}{3} F_Y(z-2)+\frac{2}{3} F_Y(z-4) \\
	p_Z(z) & =F_Z^{\prime}(z)=\frac{1}{3} p(z-2)+\frac{2}{3} p(z-4)
\end{aligned}
\]
	
	
\end{document}
