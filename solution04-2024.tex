%!TEX program = xelatex
\documentclass[14pt]{scrartcl} % A4 paper and 11pt font size
\usepackage[UTF8]{ctex}
\usepackage{lmodern}
\usepackage{amsmath}
\usepackage{amsfonts}
\usepackage{amssymb}
\usepackage[T1]{fontenc} % Use 8-bit encoding that has 256 glyphs
\usepackage{fourier} % Use the Adobe Utopia font for the document - comment this line to return to the LaTeX default
\usepackage[english]{babel} % English language/hyphenation
\usepackage{amsmath,amsfonts,amsthm} % Math packages
\usepackage{graphicx}
\usepackage{epstopdf}
\epstopdfsetup{outdir=./}
\usepackage{float}
\usepackage{lipsum} % Used for inserting dummy 'Lorem ipsum' text into the template
\usepackage{enumerate}
\usepackage{fancyvrb}
\usepackage{sectsty} % Allows customizing section commands
\allsectionsfont{\centering \normalfont\scshape} % Make all sections centered, the default font and small caps



\numberwithin{equation}{section} % Number equations within sections (i.e. 1.1, 1.2, 2.1, 2.2 instead of 1, 2, 3, 4)
\numberwithin{figure}{section} % Number figures within sections (i.e. 1.1, 1.2, 2.1, 2.2 instead of 1, 2, 3, 4)
\numberwithin{table}{section} % Number tables within sections (i.e. 1.1, 1.2, 2.1, 2.2 instead of 1, 2, 3, 4)

\setlength\parindent{0pt} % Removes all indentation from paragraphs - comment this line for an assignment with lots of text

%----------------------------------------------------------------------------------------
%	TITLE SECTION
%----------------------------------------------------------------------------------------

\newcommand{\horrule}[1]{\rule{\linewidth}{#1}} % Create horizontal rule command with 1 argument of height
\newcommand*{\Scale}[2][4]{\scalebox{#1}{$#2$}}%
\title{	
	\normalfont \huge
	\textsc{概率统计} \\ [25pt] % Your university, school and/or department name(s)
	\horrule{0.5pt} \\[0.4cm] % Thin top horizontal rule
	\huge 第四周作业答案 \\ % The assignment title
	\horrule{0.5pt} \\[0.4cm] % Thick bottom horizontal rule
	\date{}
}

\begin{document}
	\maketitle % Print the title
	10. 下列函数是随机变量 $x$ 的密度函数, 试确定常数 $a_{\circ}$ 。
	
	(1) $p(x)=a e^{-|x|},-\infty<x<\infty$;
	
	(2) $p(x)=\left\{\begin{array}{cl}a\sin \frac{x}{2}, & 0 \leqslant x \leqslant 2\pi \\ 0, & \text { 其他 }\end{array}\right.$; 
	
	(3) $p(x)=\left\{\begin{array}{cl}\cos x, & 0 \leqslant x \leqslant a \\ 0, & \text { 其他 }\end{array}\right.$; 
	
	(4) $p(x)=\left\{\begin{array}{cc}\frac{a}{1+x^{2}}, & |x| \leqslant 1 \\ 0, & |x|>1\end{array}\right.$ 。
	
	\vspace*{1cm}
	解:
	\begin{enumerate}[(1)]
		\item $\int_{-\infty}^{\infty} e ^ { -|x|}\,dx = 2\int_{0}^{\infty} e ^ {-x} \,dx = 2 \Rightarrow a = 1 / 2$
		\item $\int_{0}^{2 \pi} \sin \frac{x}{2} \,dx = 2 \int_{0}^{2 \pi} \sin \frac{x}{2} \,d\frac{x}{2} = 2 \int_{0}^{2 \pi} \,d(-\cos\frac{x}{2}) = 2 \times 2 = 4 \Rightarrow  a = \frac{1}{4}$
		\item $\int_{0}^{\pi / 2} \cos x \,dx = 1 \Rightarrow a = \frac{\pi}{2}$
		\item $\int_{-1}^{1}\frac{1}{1 + x ^ 2} \,dx = 2 \int_{0}^{1} \frac{1}{1 + x ^ 2} \,dx = 2 \int_{0}^{1} \,d \arctan x = \frac{\pi}{2} \Rightarrow a = \frac{2}{\pi}$
	\end{enumerate}
	
	14. 设 $X \sim N(5,4)$, 求 $a$, 使得 (1) $P(X<a)=0.9$; (2) $P(|X-5|>a)=0.01$ 。
	
	\vspace*{1cm}
	解:
	$\frac{X - 5}{2} \sim N(0, 1)$。
	\begin{enumerate}[(1)]
		\item $P(\frac{X - 5}{2} < \frac{a - 5}{2}) < 0.9 \Rightarrow \frac{a - 5}{2} = 1.28 \Rightarrow a = 7.56$
		\item $P(|X - 5| > a) = 0.01 \Rightarrow 2P(X - 5 < -a) =0.01 \Rightarrow P(X - 5 < -a) = 0.005 \Rightarrow P(\frac{X - 5}{2} < - \frac{a}{2}) =0.005 \Rightarrow -\frac{a}{2} = 2.576 \Rightarrow a = 5.152$
	\end{enumerate}
	
	19. 设离散型随机变量 $X$ 的分布律为
	
	\begin{tabular}{c|ccccc}
		X & -2 & -1 / 2 & 0 & 1 / 2 & 4\\
		\hline
		P & 1 / 8 & 1 / 4 & 1 / 8 & 1 / 6 & 1 / 3 \\
	\end{tabular}
	求下列随机变量的分布律: 
	
	(1) $Y=2 X ;$
	
	(2) $Y=X^{2}$; 
	
	(3) $Y=\sin \left(\frac{\pi}{2} X\right)$ 。
	
	\vspace*{1cm}
	解:
	\begin{enumerate}[(1)]
		\item 
		\begin{tabular}{c|ccccc}
			Y & -4 & -1& 0 & 1 & 8\\
			\hline
			P & 1 / 8 & 1 / 4 & 1 / 8 & 1 / 6 & 1 / 3 \\
		\end{tabular}
		\item 
		\begin{tabular}{c|cccc}
			Y & 0 & 1/4 & 4 & 16 \\
			\hline
			P & 1/8 & 5/12 & 1/8 & 1/3 \\
		\end{tabular}
		\item 
		\begin{tabular}{c|ccc}
			Y & $-\sqrt{2}/2$ & 0 & $\sqrt{2}/2$ \\
			\hline
			P & 1/4 & 7/12 & 1/6 \\
		\end{tabular}
	\end{enumerate}
	
	
\end{document}
