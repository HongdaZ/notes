\documentclass[11pt,addpoints]{exam}
\usepackage[UTF8]{ctex}
\usepackage{amsfonts,amssymb,amsmath, amsthm}
\usepackage{graphicx}
\usepackage{systeme}
\usepackage{pgf,tikz,pgfplots}
\pgfplotsset{compat=1.15}
\usepgfplotslibrary{fillbetween}
\usepackage{mathrsfs}
\usetikzlibrary{arrows}
\usetikzlibrary{calc}


\pagestyle{headandfoot}

\firstpageheader{11100200期中考试 (\numpoints\ 分)\\ 11/12/2024}{}{
	姓名: \underline{\hspace{2.5in}}\\
	学号:\underline{\hspace{2.45in}}}
%\firstpageheadrule

\runningheader{11100200期中考试}{}{Page \thepage\ of \numpages}
\runningheadrule

\firstpagefooter{}{}{}
\runningfooter{}{}{}


\begin{document}
	
	\begin{center}
		\fbox{\fbox{\parbox{6in}{\centering
					本次考试共有8道大题,其中一些有多个部分。请仔细阅读题目,写下必要的推导过程以及最后答案。你可以携带一张考前准备的A4纸。上面可以写下你认为有帮助的内容。祝你好运!
		}}}
	\end{center}
	
	\newpage
	
	
	\begin{questions} %------------------------------------------
	\question[8]
	设随机变量$X\sim N(\mu, 36), P(X < 15) = P(X > 35) = 0.048$。另外$\Phi(2) = 0.977$。试求$\mu, c$,使得$P(|X - \mu| \leq c) = 0.954$。
	
	解:
	由$P(X < 15) = P(X > 35)$可知$\mu = \frac{15 + 35}{2} = 25$。
	
	$\Phi(2) = 0.977 \Rightarrow P(\frac{|X - \mu|}{\sigma} \leq 2) = 2 \times 0.977 - 1 = 0.954$。因为$\mu = 25, \sigma = \sqrt{36} = 6$, 所以$c = 2 \times 6 = 12$。
	\vspace{\stretch{1}}
	\question
		供应商A的豆种发芽率为95\%,供应商B的豆种发芽率为70\%。一个豆种包装公司的豆种30\%来自供应商A,70\%来自供应商B。并且全部豆种被均匀混合。
		\begin{parts}
			\part[4]
			计算从混合好的豆种中随机抽取的一粒可以发芽的概率$P(G)$.\\
			解:\\
			$P(G)=P(A) \cdot P(G \mid A)+P(B) \cdot P(G \mid B)=0.3 \times 0.95+0.7 \times 0.7=0.775$
			
			\vspace{\stretch{1}}
			\part[4]
			给定一粒被抽取的种子可以发芽,计算它来自供应商A的条件概率。\\
			解:
			$P(A \mid G)=\dfrac{P(A \cap G)}{P(G)}=\dfrac{P(G \mid A) \cdot P(A)}{P(G)}=\dfrac{0.285}{0.775}=0.37$
			\vspace{\stretch{2}}
		\end{parts}
		
		\newpage
		
	 	\question
	 	一个教授给她的学生们出了15道作业题,其中4道将被当作期中考试题。一个学生只有时间学习15道题中的5道。请计算如下概率
	 	\begin{parts}
	 		\part[4]
	 		学生学习到了3道期中考试题。\\
	 		解:$\dfrac{C_4^3C_{11}^2}{C_{15}^5} = 0.073$.
	 		\vspace{\stretch{1}}
	 		\part[4]
	 		至少学习到了1道期中考试题。\\
	 		解:$1 - \dfrac{C_{11}^5}{C_{15}^5} = 0.8462$
	 		\vspace{\stretch{1}}
	 	\end{parts}
	 	
	 	\newpage
	 	
	 	\question
	 	对于下面每个函数,(i)计算常量$c$使得$f(x)$是一个随机变量$X$的密度函数,(ii)计算概率分布函数$F(x) = P(X \leq x)$,(iii)计算$X$均值$\mu$和方差$\sigma ^ 2$
	 	
	 	\begin{parts}
	 		\part[10]
	 		$f(x) = (3 / 16)x ^ 2 , -c < x < c$。\\
	 		解:	\\
	 		(i) $\int_{-c}^{c}(3 / 16) x^{2} d x=1 \Leftrightarrow 	c^{3} / 8 =1  \Leftrightarrow c =2$
	 		
	 		
	 		(ii) $F(x)=\int_{-\infty}^{x} f(t) d t	=\int_{-2}^{x}(3 / 16) t^{2} d t=\dfrac{t^{3}}{16}\rvert_{-2}^{x}=\dfrac{x^{3}}{16}+\dfrac{1}{2}$
	 		
	 		
	 		$$
	 		F(x)= \begin{cases}0, & -\infty<x<-2 \\ \dfrac{x^{3}}{16}+\dfrac{1}{2}, & -2 \leq x<2 \\ 1, & 2 \leq x<\infty\end{cases}
	 		$$
	 		(iii)
	 		$$
	 		\begin{aligned}
	 			\mu & =\int_{-2}^{2}(x)(3 / 16)\left(x^{2}\right) d x=0 \\
	 			\sigma^{2} & =\int_{-2}^{2}\left(x^{2}\right)(3 / 16)\left(x^{2}\right) d x=\dfrac{12}{5} .
	 		\end{aligned}
	 		$$
	 		\vspace{\stretch{1}}
	 		\newpage
	 		\part[10]
	 		$f(x) = c / \sqrt{x}, 0 < x < 1$。\\
	 		解:
	 		(i) $\int_{0}^{1} \dfrac{c}{\sqrt{x}} d x =1 \Leftrightarrow 2 c  =1 \Leftrightarrow c=1 / 2$
	 		
			(ii)
	 		$$\begin{aligned}
	 			F(x) & =\int_{-\infty}^{x} f(t) d t \\
	 			& =\int_{0}^{x} \dfrac{1}{2 \sqrt{t}} d t \\
	 			& =\sqrt{t}|_{0}^{x}=\sqrt{x}, \\
	 			F(x) & = \begin{cases}0, & -\infty<x<0 \\
	 				\sqrt{x}, & 0 \leq x<1, \\
	 				1, & 1 \leq x<\infty .\end{cases}
	 		\end{aligned}
	 		$$
	 		(iii)
	 		$$
	 		\begin{aligned}
	 			\mu & =\int_{0}^{1}(x)(1 / 2) / \sqrt{x} d x=\dfrac{1}{3} \\
	 			E\left(X^{2}\right) & =\int_{0}^{1}\left(x^{2}\right)(1 / 2) / \sqrt{x} d x=\dfrac{1}{5} \\
	 			\sigma^{2} & =\dfrac{1}{5}-\left(\dfrac{1}{3}\right)^{2}=\dfrac{4}{45}
	 		\end{aligned}
	 		$$
	 	\end{parts}
	 	
	 	\question
	 	投掷两枚正四面体骰子,一个红色,一个黑色。每一个骰子等可能出现1、2、3、4四种结果。用$X$代表红色骰子出现的结果,$Y$代表两个骰子\textbf{结果之和}。
	 	\begin{parts}
	 		\part[5]
	 		计算$X,Y$的联合分布列。\\
	 		解:$f(x, y) = 1 / 16, x = 1, 2, 3, 4; y = x + 1, x + 2, x + 3, x +4$
	 		\vspace{\stretch{2}}
	 		\part[5]
	 		计算$X$的边缘分布列。\\
	 		解:$f_X(x) = 1 / 4, x = 1, 2, 3, 4$
	 		\vspace{\stretch{1}}
	 		\part[5]
	 		计算$Y$的边缘分布列。\\
	 		解:$f_Y(y) = (4 − |y − 5|)/16, y = 2, 3, 4, 5, 6, 7, 8;$
	 		\vspace{\stretch{1}}
	 		\part[5]
	 		请说明$X, Y$是否独立,及其原因。\\
	 		解:不独立。因为$Y$取值范围与$X$有关。
	 		\vspace{\stretch{1}}
	 	\end{parts}
	 	\newpage
	 	\question
	 	$X, Y$有联合密度函数$f(x, y) = x + y, 0 \leq x \leq 1, 0 \leq y \leq 1$
	 	\begin{parts}
	 		\part[5]
	 		计算边缘密度函数$f_X(x), f_Y(y)$。\\
	 		解:
	 		 $\quad f_{X}(x)=\int_{0}^{1}(x+y) d y	=(x y+\dfrac{1}{2} y^{2})\bigg|_{0}^{1}=x+\dfrac{1}{2}, \quad 0 \leq x \leq 1$
	 		
	 		$f_{Y}(y)=\int_{0}^{1}(x+y) d x=y+\dfrac{1}{2}, \quad 0 \leq y \leq 1 ;$
	 		\vspace{\stretch{1}}
	 		\part[5]
	 		随机变量$X, Y$是否独立?请解释原因。\\
	 		解:
	 		否。
	 		$f(x, y)=x+y \neq\left(x+\dfrac{1}{2}\right)\left(y+\dfrac{1}{2}\right)=f_{X}(x) f_{Y}(y)$.
	 		\vspace{\stretch{0.5}}
	 		\part[5]
	 		计算$X, Y$的均值和方差。\\
	 		解:
	 		
	 		(i) $\mu_{X}=\int_{0}^{1}x\left(x+\dfrac{1}{2}\right) d x=(\dfrac{1}{3} x^{3}+\dfrac{1}{4} x^{2})\bigg|_{0}^{1}=\dfrac{7}{12}$; \\
	 		(ii) $\mu_{Y}=\int_{0}^{1} y\left(y+\dfrac{1}{2}\right) d y=\dfrac{7}{12}$;\\
	 		(iii) $E\left(X^{2}\right)=\int_{0}^{1} x^{2}\left(x+\dfrac{1}{2}\right) d x=(\dfrac{1}{4} x^{4}+\dfrac{1}{6} x^{3})\bigg|_{0}^{1}=\dfrac{5}{12}$,
	 		
	 		$
	 		\sigma_{X}^{2}=E\left(X^{2}\right)-\mu_{X}^{2}=\dfrac{5}{12}-\left(\dfrac{7}{12}\right)^{2}=\dfrac{11}{144} .
	 		$
	 		
	 		(iv) Similarly, $\sigma_{Y}^{2}=\dfrac{11}{144}$.
	 		\vspace{\stretch{1}}
	 		\part[5]
	 		计算$P(X \leq Y)$ \\
	 		解:$P(X \leq Y) = 1 / 2$
	 		\vspace{\stretch{1}}
	 		
	 		\part[5]
	 		计算$Z = X + Y$的密度函数。\\
	 		解:
	 		
	 		$\begin{aligned} & 0 \leq z \leq 1 \\ & F_Z(z)=\int_0^z \int_0^{z-x} x+y d y d x \\ &=\frac{1}{3} z^3 \\ & p_Z(z)=F_Z^{\prime}(z)=\frac{1}{3} \times 3 z^2=z^2 \\ & 1<z \leq 2 \\ & F_Z(z)=1-\int_{z-1}^1 \int_{z-x}^1 x+y d y d x \\ &=-\frac{z^3-3 z^2+1}{3} \\ & p_Z(z)=F_z^{\prime}(z)=-z^2+2 z \\ & \text { 综上 } \quad p_Z(z)=\left\{\begin{array}{cc}z^2, & z \in[0,1] \\ -z^2+2 z, & z \in(1,2] \\ 0 & \text { 其它 }\end{array}\right.\end{aligned}$
	 	\end{parts}
	 	\newpage
	 	\question[10]
	 	$X$有密度函数$f(x) = x\exp(-x ^ 2 / 2), 0 < x < \infty. Y = X ^ 2$。计算$Y$的密度函数。\\
	 	解:
	 	$x=\sqrt{y}, \dfrac{d x}{d y}=\dfrac{1}{2 \sqrt{y}}$ 并且 $0<x<\infty$ 映射到 $0<y<\infty$. 所以
	 	
	 	$$
	 	g(y)=\sqrt{y}\exp(-y / 2)\left|\dfrac{1}{2 \sqrt{y}}\right|=\dfrac{1}{2} e^{-y / 2}, \quad 0<y<\infty
	 	$$
	 	\vspace{\stretch{1}}
	 	\question
	 	随机变量$X, Y$有联合分布列$f(x, y) = 1 / 4, (x, y) \in S = \{(0, 0), (1, 1), (1, -1), (2, 0)\}$
	 	\begin{parts}
	 		\part[5]
	 		$X, Y$是否独立?请解释原因。 \\
	 		解:否。$Y$的取值范围由$X$决定
	 		\vspace{\stretch{1}}
	 		\part[5]
	 		计算协方差和相关系数。\\
	 		解: $\operatorname{Cov}(X, Y) = \rho = 0$
	 		\vspace{\stretch{1}}
	 	\end{parts}
	\end{questions}
	
\end{document}